
\chapter{Availability and Compilation}
\label{sec:availability_compilation}
\section{Availability}
\label{sec:Availability}
\index{repository}\index{availability}
This software is available on GitHub:
\begin{center}
	\url{https://github.com/jkriege2/FCSSimulator}
\end{center}
In addition this simulator is also a part (based on the same repository) of the FCS/FCCS data evaluation software \qf (\url{https://github.com/jkriege2/QuickFit3} and \url{http://www.dkfz.de/Macromol/quickfit/}), which also offers an integrated editor-GUI for the simulator configuration scripts. Therefore if you don't want to compile \df for yourself, or modify its source-code you will only have to install \qf and can skip the following sections~\ref{sec:DownloadAndSetup}~\&~\ref{sec:Compilation}. You can even use \qf to edit the \df configuration files and run the simulation, so also sections~\ref{sec:RunningTheSoftware}~\&~\ref{sec:RunningDfFromAMakefile} are only of minor interest! Also \qf can be used to easily read the FCS/FCCS-correlation curves from \df and evaluate them further. However, additional results might be created by \df. These are usually saved in standard ASCII- or CSV-files and can be viewed and evaluated with any standard data-evaluation software. But we propose to use the open-source tools described in section~\ref{sec:UsefullAdditionalTools}, as \df is optimized to work together with them!

\section{Download and Setup of \df for COmpilation}
\label{sec:DownloadAndSetup}
\index{Download}\index{build environment}\index{build}
\df is a C++ program, so you will need a working C++ compiler (e.g. the \textsc{GCC} in version $\geq$4.7, for Windows you can use \textsc{MinGW-Builds} from \url{http://sourceforge.net/projects/mingwbuilds/}). The program only relies on the GNU scientific library (GSL), available from \url{http://www.gnu.org/software/gsl/}, but also integrated in the repository (see below). In addition you will need a working \bash-shell (on Windows use \msys available from \url{http://www.mingw.org/wiki/msys}) and \textsc{Git} to access the repository (if you don't want to download the code manually). In general this program should compile on \textsc{Windows}, \textsc{Linux} and \textsc{MacOS X}.\\[10mm]

\noindent In order to install \df, follow these steps:
\begin{enumerate}
	\item check out the git repository: 
	  \begin{lstlisting}[language=bash] 
$ git clone --recursive "https://github.com/jkriege2/FCSSimulator.git"
		\end{lstlisting}
	\item ensure that all submodules are checked out: 
	  \begin{lstlisting}[language=bash] 
$ cd FCSSimulator
$ git submodule update --init --remote --recursive --force
		\end{lstlisting}
	\item if GSL is not available on your system, we'll have to build it now. The repository contains a \bash-script for this purpose: 
	  \begin{lstlisting}[language=bash] 
$ cd extlibs
$ . build_dependencies.sh
$ cd..
		\end{lstlisting}
		This script will build a \underline{local version} of \GSL. It is not installed in the system, but only in the directory \texttt{./extlibs/gsl/}. It will ask several questions: Generally you should not keep the build-directories (\texttt{n}), use as many processor, as you like (e.g. \texttt{2} for a dual-core, or \texttt{4} for a quad-core machine), you the best optimizations for your local machine, if the programm will only be used locally (twice \texttt{y}, or for safe-settings \texttt{y} and then \texttt{n}). Finally answer \texttt{y} when asked whether GSL should be compiled.
\end{enumerate}

\section{Compilation}
\label{sec:Compilation}
\index{compilation}
the rest of the compilation is done using a \texttt{Makefile} in the base directory:
\begin{lstlisting}[language=bash] 
$ make 
\end{lstlisting}

\section{Running the Software \df}
\label{sec:RunningTheSoftware}
\index{running}
After the compilation, an command-line based executable \texttt{diffusion4} is created in the base directory. This is the simulator, which can then be started as follows:
\begin{lstlisting}[language=bash] 
$ diffusion4 [--spectra SPECTRADIRECTORY] CONFIG_SCRIPT.ini 
\end{lstlisting}
Here \texttt{CONFIG\_SCRIPT.ini} is a configuration file that tells the simulator what to do and the optional \texttt{--spectra SPECTRADIRECTORY} can be used to provide another directory with absorption and emission spectra. A version of this directory is available in the repository under \verb!./spectra/!, which is also loaded automatically, if the option \verb!--spectra! is not given. \df reads absorption and emission spectra for its simulation from this directory. Note that \df will not work, if a \texttt{spectra}-directory is not provided!

\section{Running \df from a Makefile}
\label{sec:RunningDfFromAMakefile}
\index{running!Makefile}
Often it is useful to run \df from a Makefile, if several simulations should be done (in parallel) on a computer. here is an example Makefile for this task (see the repository directory \texttt{./example\_configs/} for further examples:
\begin{lstlisting}[language=make] 
SCRIPTS= gyuri_isat_multigfp0.ini\
	       gyuri_isat_multigfp1.ini 

SHELL = sh

SCRIPTS_TARGET = $(subst .ini,.target,$(SCRIPTS))

TERMINAL_COMMAND=		

ifeq ($(findstring Msys,$(OS)),Msys)
EXE_SUFFIX=.exe
TERMINAL_COMMAND=
else
EXE_SUFFIX=
#TERMINAL_COMMAND=konsole -e 
#TERMINAL_COMMAND=x-terminal-emulator -e 
endif		
		 
all: ${SCRIPTS_TARGET}

%.target: %.ini
	@echo -e "random wait before starting $< ..."
	@bash random_sleep.sh
	@bash -c "sleep $$[ ( $$RANDOM % 10 )  + 1 ]s"
	@echo -e "starting on $< ..."
	${TERMINAL_COMMAND} ./diffusion4${EXE_SUFFIX} $< > $<.log
	@echo -e "work on $< DONE!"
\end{lstlisting}
You can run such a Makefile with the command:
\begin{lstlisting}[language=bash] 
$ make -f Makefile -j4
\end{lstlisting}
where \texttt{4} specifies the number of processors to use. This makefile also uses the \bash-script \texttt{random\_sleep.sh}, which waits a random number of seconds (up to 10 or 20) before running  the simulation. This ensures that the random number generator of each simulation is initialized with a different seed (the seed is taken from the system time!).

\section{Usefull Additional Tools}
\label{sec:UsefullAdditionalTools}
\df will store its result in a bunch of standardized output-files. Many modules in \df generate data-files in the comma-separated values (CSV) format, or in other standardized data-formats (e.g. TIFF for images, or ALV-5000 ASCII-files for correlation data). Therefore you can use any standard data-evaluation software to further process the simulation results. Nevertheless, \df was optimized to work well with these open-source softwares:
\begin{itemize}
	\item \textsc{Gnuplot 4.6}\index{Gnuplot} is a data-plotting and evaluation tool. It is available for most platforms from 
					\begin{center}
						\url{http://gnuplot.info/}.
					\end{center}
				Many modules in \df generate \textsc{Gnuplot}-files (\texttt{*.plt})\index{.plt} in addition to CSV output files. These \textsc{Gnuplot}-files allow to immediately generate plots from the data, output by \df. Usually they are generated such that they first generate a PDF of the plots and then display them on the screen.
	\item \textsc{GraphViz}\index{GraphViz} is used to plot graphs, that e.g. represent the structure of a \df-simulation. This software is available from
						\begin{center}
						\url{http://www.graphviz.org/}.
					\end{center}
     \df e.g. generates \texttt{.gv}-files that can be run through the \textsc{GraphViz}-module \texttt{dot} and then output a graph representing all generator- and detector-objects in the \df-simulation and their relation to each other.
\end{itemize}
